% autosam.tex
% Annotated sample file for the preparation of LaTeX files
% for the final versions of papers submitted to or accepted for
% publication in AUTOMATICA.

% See also the Information for Authors.

% Make sure that the zip file that you send contains all the
% files, including the files for the figures and the bib file.

% Output produced with the elsart style file does not imitate the
% AUTOMATICA style. The style file is generic for all Elsevier
% journals and the output is laid out for easy copy editing. The
% final document is produced from the source file in the
% AUTOMATICA style at Elsevier.

% You may use the style file autart.cls to obtain a two-column
% document (see below) that more or less imitates the printed
% Automatica style. This may helpful to improve the formatting
% of the equations, tables and figures, and also serves to check
% whether the paper satisfies the length requirements.

% Please note: Authors must not create their own macros.

% For further information regarding the preparation of LaTeX files
% for Elsevier, please refer to the "Full Instructions to Authors"
% from Elsevier's anonymous ftp server on ftp.elsevier.nl in the
% directory pub/styles, or from the internet (CTAN sites) on
% ftp.shsu.edu, ftp.dante.de and ftp.tex.ac.uk in the directory
% tex-archive/macros/latex/contrib/supported/elsevier.


%\documentclass{elsart}               % The use of LaTeX2e is preferred.

\documentclass[twocolumn]{autart}    % Enable this line and disable the
                                     % preceding line to obtain a two-column
                                     % document whose style resembles the
                                     % printed Automatica style.

\usepackage{color}
\newcommand{\comment}[1]{\textcolor{red}{\textbf{[#1]}}}

\usepackage{amsmath,amsfonts}
\usepackage{graphicx}          % Include this line if your
                               % document contains figures,
%\usepackage[dvips]{epsfig}    % or this line, depending on which
                               % you prefer.

\newtheorem{algo}{Algorithm}[section]

\begin{document}

\begin{frontmatter}
%\runtitle{Insert a suggested running title}  % Running title for regular
                                              % papers but only if the title
                                              % is over 5 words. Running title
                                              % is not shown in output.

%\title{In Catilinam IV\thanksref{footnoteinfo}} % Title, preferably not more
                                                % than 10 words.

%\thanks[footnoteinfo]{This paper was not presented at any IFAC
%meeting. Corresponding author M.~T.~Cicero. Tel. +XXXIX-VI-mmmxxi.
%Fax +XXXIX-VI-mmmxxv.}

%\author[Paestum]{Marcus Tullius Cicero}\ead{cicero@senate.ir},    % Add the
%\author[Rome]{Julius Caesar}\ead{julius@caesar.ir},               % e-mail address
%\author[Baiae]{Publius Maro Vergilius}\ead{vergilius@culture.ir}  % (ead) as shown

%\address[Paestum]{Buckingham Palace, Paestum}  % Please supply
%\address[Rome]{Senate House, Rome}             % full addresses
%\address[Baiae]{The White House, Baiae}        % here.


%\begin{keyword}                           % Five to ten keywords,
%Cicero; Catiline; orations.               % chosen from the IFAC
%\end{keyword}                             % keyword list or with the
                                          % help of the Automatica
                                          % keyword wizard

\title{Rao-Blackwellised Particle Smoothing for
Conditionally Linear Gaussian Models}

%\thanksref{footnoteinfo}
%\thanks[footnoteinfo]{This work was supported in part by the XXX.}

\author[First]{Simo S\"arkk\"a}
\author[Second]{Pete Bunch}
\author[Third]{Simon J. Godsill}

\address[First]{Aalto University, P.O. Box
  12200. FI-00076 AALTO, Finland. (Tel: +358 50 512 4393; e-mail: simo.sarkka@aalto.fi)}
\address[Second]{University of Cambridge, Department of Engineering, Trumpington Street, Cambridge CB2 1PZ, UK (e-mail: pb404@cam.ac.uk)}
\address[Third]{University of Cambridge, Department of Engineering, Trumpington Street, Cambridge CB2 1PZ, UK (e-mail: sjg@cam.ac.uk)}


\begin{keyword}
estimation algorithms; optimal estimation; nonlinear systems;
Monte Carlo method; stochastic systems.
\end{keyword}

\begin{abstract}  % Abstract of not more than 200 words.
Abstract is here.
\end{abstract}

\end{frontmatter}

\section{Introduction}
%
\comment{Something about particle filtering and its efficiency as well asmotivation of Rao-Blackwellisation}

\comment{Comment something about the conference article
  \cite{Sarkka+Bunch+Godsill:2012}}

\comment{Contributions of this article here}

The structure of the paper is as follows: \ldots

%\begin{figure}
%\begin{center}
%\includegraphics[height=4cm]{jcaesar.eps}    % The printed column
%\caption{Gaius Julius Caesar, 100--44 B.C.}  % width is 8.4 cm.
%\label{fig1}                                 % Size the figures
%\end{center}                                 % accordingly.
%\end{figure}

\subsection{Problem Formulation}
%
\comment{I think we should maybe put the Bayesian Filtering and Smoothing section before the Problem Formulation}
%
This paper is concerned with Rao-Blackwellised particle smoothing of conditionally linear Gaussian hidden Markov models
(see \cite{Doucet+Godsill+Andrieu:2000}) of the form,
%
\begin{equation}
\begin{split}
  u_k &\sim p(u_k \, \,|\, \, u_{k-1}) \\
  z_k &= A(u_{k}) \, z_{k-1} + q_{k} \\
  y_k &= H(u_k) \, z_{k} + r_{k},
\end{split}
\label{eq:condgauss}
\end{equation}
%
where $y_k$ for $k=1,\ldots,M$ are the measurements and the state consists of two parts $x_k = (u_k,z_k)$, where $z_k \in \mathbb{R}^{n_z}$ is conditionally Gaussian given the non-Gaussian part $u_k \in \mathbb{R}^{n_u}$. The Gaussian process noise $q_{k} \sim \mathrm{N}(0, Q(u_{k}))$ and observation noise $r_{k} \sim \mathrm{N}(0, R(u_{k}))$ are assumed to be time-white. Here we formally treat the variable $u_k$ as having a probability density with respect to the Lebesgue measure, but all the results apply or can be easily generalised to random variables without such densities.

In particular, we consider smoothers which do not require sampling of the linear portion of the state $z_k$. This feature is required for the smoother to be truly Rao-Blackwellised (marginalised) as opposed to, for example, smoothers presented by \cite{Fong+Godsill+Doucet+West:2002} and \cite{Lindsten+Schon:2011}, which also sample the linear part.

\subsection{Bayesian Filtering and Smoothing}
%
The theory of optimal Bayesian filtering and smoothing \cite{Ho+Lee:1964,Lee:1964} addresses the problem of state estimation in probabilistic state space models of the form,
%
\begin{equation}
\begin{split}
  x_k &\sim p(x_k \, \,|\, \, x_{k-1}) \\
  y_k &\sim p(y_k \, \,|\, \, x_{k}),
\end{split}
\label{eq:genmodel}
\end{equation}
%
where $k=1,\ldots,M$, $x_k \in \mathbb{R}^{d_x}$ is the state of the system, $y_k \in \mathbb{R}^{d_y}$ is the observation at time step $k$, $p(x_k \, \,|\, \, x_{k-1})$ is the Markovian dynamic model probability density and $p(y_k \, \,|\, \, x_{k})$ is the measurement model probability density.

The formal {\em Bayesian filtering} solution to this state space model is given by the well-known recursions \cite{Ho+Lee:1964},
%
\begin{equation}
\begin{split}
   p(x_k \,|\, y_{1:k-1})
   &= \int p(x_k \,|\, x_{k-1}) \, p(x_{k-1} \,|\, y_{1:k-1}) \, dx_{k-1} \\
   p(x_k \,|\, y_{1:k})
   &\propto
   p(y_k \,|\, x_k,y_{1:k-1}) \, p(x_k \,|\, y_{1:k-1}).
\end{split}
\label{eq:bfilter}
\end{equation}
%
In the linear Gaussian case, these equations have a closed form solution, which leads to the celebrated Kalman filter \cite{Kalman:1960}.

The {\em Bayesian smoothing} solutions to such probabilistic state space models can be divided into (1) two-filter smoothers (2) forward-backward smoothers:
%
\begin{enumerate}
\item {\em The two-filter} smoother \cite{Lee:1964,Kitagawa:1994} can be interpreted as a combination of forward and backward Bayesian filters,
%
\begin{equation}
\begin{split}
  p(x_k \,|\, y_{1:M})
  &\propto
  p(x_k \,|\, y_{1:k}) \, p(y_{k+1:M} \,|\, x_k),
\end{split}
%\nonumber
\end{equation}

where $p(x_k \,|\, y_{1:k})$ is computed by an ordinary forward filter, and $p(y_{k+1:M} \,|\, x_k)$ by an unnormalised filter using the backward recursion,
%
\begin{equation}
\begin{split}
  p(y_{k+1:M} \,|\, x_k)
  &= \int p(y_{k+1:M} \,|\, x_{k+1}) \, p(x_{k+1} \,|\, x_k) \,
          dx_{k+1} \\
  p(y_{k:M} \,|\, x_k)
  &= p(y_k \,|\, x_k) \, p(y_{k+1:M} \,|\, x_k).
\end{split}
%\nonumber
\label{eq:tfsmooth}
\end{equation}

In the linear case we get a combination of two Kalman filters \cite{Fraser+Potter:1969}, one going forward and one backward in time. For numerical reasons, the backward Kalman filter is usually convenient to implement as an information filter. \comment{Cite something?}

\item {\em The forward-filtering backward-smoothing} algorithm   \cite{Kitagawa:1987} is based on a backward sweep, which uses the filter results,
%
\begin{equation}
\begin{split}
 &p(x_{k} \,|\, y_{1:M})
  = p(x_{k}\,|\,y_{1:k}) \\
 &\qquad \times \int
    \frac{p(x_{k+1} \,|\, y_{1:M}) \, p(x_{k+1}\,|\,x_{k})}
         {p(x_{k+1}\,|\,y_{1:k})} \, dx_{k+1}.
\end{split}
%\nonumber
\label{eq:fbsmooth}
\end{equation}
%
In the Gaussian case this smoother reduces to the Rauch-Tung-Striebel (RTS) smoother \cite{Rauch+Tung+Striebel:1965}.
\end{enumerate}

\subsection{Particle Smoothing}
%
As the Bayesian filtering equations do not in general admit closed form solutions, approximations are needed. A general class of these is the particle filter \cite{Gordon+Salmon+Smith:1993,Kitagawa:1996,Doucet+Godsill+Andrieu:2000,Ristic+Arulampalam+Gordon:2004,Cappe+Godsill+Moulines:2007}, which generates a Monte Carlo approximation to the filtering distribution using the sequential importance resampling (SIR) algorithm. Formally this means the following approximation to the probability density, which consists of a weighted sum of Dirac delta functions:
%
\begin{equation}
  p(x_k\,|\,y_{1:k}) \approx \sum_i w_k^{(i)} \delta(x_k - x_k^{(i)}).
\label{eq:pfapp}
\end{equation}
%
%The generic SIR particle filter works as follows:
%%
%\begin{algo}[Particle filter]
%  Given the state space model \eqref{eq:genmodel}, importance
%  density $q(x_{k} \, | \, x_{k-1},y_{1:k})$ and measurements $\{ y_k
%  \}$:
%\begin{itemize}
%\item Start from the set $\{ w^{(i)}_0,x_0^{(i)} ~|~
%  i=1,\ldots,N \}$ of importance samples representing the
%  prior distribution.
%\item For $k = 1,\ldots,M$:
%\begin{enumerate}
%\item For each $i=1,\ldots,N$ draw a sample from the importance
%  distribution
%  \begin{equation}
%     x_k^{(i)} \sim q(x_{k} \, | \, x_{k-1}^{(i)},y_{1:k}),
%  \end{equation}
%
%\item Calculate new weights according to
%  \begin{equation}
%    w_k^{(i)} \propto
%    w_{k-1}^{(i)} \frac{p(y_k \, | x_k^{(i)}) \, p(x_k^{(i)}\,|\,x_{k-1}^{(i)})}
%                       {q(x_{k}^{(i)} \, | \, x_{k-1}^{(i)},y_{1:k})}
%  \end{equation}
%%
%  and normalize them to sum to unity.
%
%\item If the effective number of particles is too low, resample the
%  set $\{ w_k^{(i)}, x_{1:k}^{(i)} \}$ by interpreting each weight
%  $w_k^{(i)}$ as the probability of obtaining the sample index $i$ in
%  the set $\{ x_k^{(i)} ~|~i=1,\ldots,N \}$.
%\end{enumerate}
%
%\item At step $k$, the approximation to the filtering density is then
%  given by Equation \eqref{eq:pfapp}.
%\end{itemize}
%\end{algo}
%
%\subsection{Particle Smoothing}

Analogously to particle filters, particle smoothers can be used to form approximate solutions to the Bayesian smoothing equations, which also do not admit closed form solutions in general. The most basic particle smoother of \cite{Kitagawa:1996} is a trivial extension to the particle filter---if we simply store the entire sample histories and resample these instead of single-step samples, we obtain a smoothing solution alongside the filtering solution.

The generic SIR particle filter/smoother \cite{Kitagawa:1996} works as follows.
%
\begin{algo}[Basic particle filter/smoother]
  Given the state space model \eqref{eq:genmodel}, importance
  density $q(x_{k} \, | \, x_{k-1},y_{1:k})$ and measurements $\{ y_k
  \}$:
\begin{itemize}
\item Start from the set $\{ w^{(i)}_0,x_0^{(i)} ~|~
  i=1,\ldots,N \}$ of importance samples representing the
  prior distribution.
\item For $k = 1,\ldots,M$:
\begin{enumerate}
\item For each $i=1,\ldots,N$ draw a sample from the importance
  distribution
  \begin{equation}
     x_k^{(i)} \sim q(x_{k} \, | \, x_{k-1}^{(i)},y_{1:k}),
  \end{equation}
%
  and append the samples to the state histories:
%
  \begin{equation}
     x_{0:k}^{(i)} = (x_{0:k-1}^{(i)},x_k^{(i)}).
  \end{equation}
%
\item Calculate new weights according to
  \begin{equation}
    w_k^{(i)} \propto
    w_{k-1}^{(i)} \frac{p(y_k \, | x_k^{(i)}) \, p(x_k^{(i)}\,|\,x_{k-1}^{(i)})}
                       {q(x_{k}^{(i)} \, | \, x_{k-1}^{(i)},y_{1:k})}
  \end{equation}
%
  and normalize them to sum to unity.

\item If the effective number of particles is too low, resample the
  history set $\{ w_k^{(i)}, x_{0:k}^{(i)} \}$.
\end{enumerate}

\item At step $k$, the approximation to the smoothing density is
  then given by
  \begin{equation}
     p(x_{1:k}\,|\,y_{1:k})
     \approx \sum_i w_k^{(i)} \delta(x_{1:k} - x_{1:k}^{(i)}),
  \end{equation}
  %
  and an approximation to the filtering density can be obtained by
  discarding the states $1,\ldots,k-1$, i.e. by Equation \eqref{eq:pfapp}.
\end{itemize}
\end{algo}
%
The problem with this approach is that its estimate of the smoothing distribution tends to be degenerate \cite{Kitagawa:1996}, with few or only one unique particles throughout most of the history. For this reason, various alternative particle smoothers have been proposed. One possible approach is to approximate the two-filter formula using particle methods \cite{Kitagawa:1996,Briers+Doucet+Maskell:2010,Fearnhead:2010}. The disadvantage of such two-filter type smoothers is that the construction of the backward filter can be quite troublesome. \comment{Is it really? Briers \& co seem to have done it in their 2010 paper.} Another way is to compute new weights for the particle filter results using an additional backward sweep, as is done in particle smoothers of \cite{Hurzeler+Kunsch:1998} and \cite{Doucet+Godsill+Andrieu:2000}.
%Because these approaches only approximate the marginal smoothing
%distributions without generating explicit trajectories from the joint
%smoothing distribution, their usage in Rao-Blackwellized particle
%smoothing is not straight-forward.

The {\em backward-simulation particle smoother} of \cite{Godsill+Doucet+West:2004} is based on simulation of individual trajectories backwards, starting from the last step and proceeding to the first. Assume that we have already simulated a trajectory $\tilde{x}_{k+1:M}$ from the smoothing distribution. Because of the Markovianity of the model, we get,
%
\begin{equation}
\begin{split}
  p(x_{k}\,|\,\tilde{x}_{k+1},y_{1:M})
  &= p(x_{k}\,|\,\tilde{x}_{k+1},y_{1:k}) \\
  &= \frac{p(\tilde{x}_{k+1}\,|\,x_{k},y_{1:k}) \, p(\tilde{x}_{k}\,|\,y_{1:k})}
          {p(\tilde{x}_{k+1}\,|\,y_{1:k})} \\
  &= \frac{p(\tilde{x}_{k+1}\,|\,x_{k}) \, p(\tilde{x}_{k}\,|\,y_{1:k})}
          {p(\tilde{x}_{k+1}\,|\,y_{1:k})} \\
  &\propto \, p(\tilde{x}_{k+1}\,|\,x_{k}) \, p(x_{k}\,|\,y_{1:k}).
\end{split}
\end{equation}
%
By substituting the SIR filter approximation in Equation \eqref{eq:pfapp}, we get,
%
\begin{equation}
\begin{split}
   p(x_{k}\,|\,\tilde{x}_{k+1},y_{1:M})
   &\propto \sum_i w_k^{(i)} \, p(\tilde{x}_{k+1}\,|\,x_{k}^{(i)}) \,
   \delta(x_k - x_k^{(i)}).
\end{split}
\label{eq:backsimeq}
\end{equation}
%
We can now draw a sample from this distribution by selecting $x_{k}^{(i)}$ with probability $\propto w_k^{(i)} p(\tilde{x}_{k+1}\,|\,x_{k}^{(i)})$. The resulting algorithm is as follows.
%
\begin{algo}[Backward-simulation particle smoother]
  \label{alg:bssmooth} \index{Backward-simulation particle smoother}
  Given the weighted set of particles $\{ w_k^{(i)}, x_k^{(i)} \,|\,
  i=1,\ldots,N,~k=1,\ldots,M \}$ representing the filtering
  distributions:
\begin{itemize}
\item Choose $\tilde{x}_M = x_M^{(i)}$ with probability $w^{(i)}_{M}$.
\item For $k=M-1,\ldots,0$:
\begin{enumerate}
  \item Compute new weights by
    \begin{equation}
      \begin{split}
        w^{(i)}_{k|k+1} \propto
        w_k^{(i)} p(\tilde{x}_{k+1}\,|\,x_{k}^{(i)}).
     \end{split}
    \end{equation}
  \item Choose $\tilde{x}_k = x_k^{(i)}$ with probability
    $w^{(i)}_{k|k+1}$.
  \end{enumerate}
\end{itemize}
\end{algo}
%
Given $S$ iterations of the above procedure resulting in sample trajectories $\tilde{x}_{1:M}^{(j)}$ for $j=1,\ldots,S$ the smoothing distribution can now be approximated as,
%
\begin{equation}
     p(x_{1:M}\,|\,y_{1:M}) \approx
      \frac{1}{S} \sum_j \delta(x_{1:M} - \tilde{x}_{1:M}^{(j)}).
\end{equation}
%
The marginal distribution samples for a step $k$ can be obtained by extracting the $k$th components from the above trajectories. The computational complexity of the method is $O(S \, M \, N)$. However, the result is much less degenerate than of the particle smoother of the filter/smoother.

The {\em reweighting particle smoother} of \cite{Hurzeler+Kunsch:1998} and \cite{Doucet+Godsill+Andrieu:2000} is based on computing new weights for the SIR filtering particles such that we get an approximation to the smoothing distribution. Assume that we have already computed the weights for the following approximation, where the particles $x_{k+1}^{(i)}$ are from the filter,
%
\begin{equation}
\begin{split}
   p(x_{k+1}\,|\,y_{1:M})
   \approx
   \sum_i w^{(i)}_{k+1|M} \, \delta(x_{k+1} - x_{k+1}^{(i)}).
\end{split}
\label{eq:rwassume}
\end{equation}
%
By substituting the dynamic model to the filter approximation we get the following approximation to the predicted distribution,
%
\begin{equation}
\begin{split}
  p(x_{k+1}\,|\,y_{1:k})
  &\approx \sum_j w_k^{(j)} \, p(x_{k+1}\,|\,x_{k}^{(j)}).
\end{split}
\label{eq:rwpred}
\end{equation}
%
Substituting Equations \eqref{eq:rwassume} and \eqref{eq:rwpred} into the integral in Equation \eqref{eq:fbsmooth} results in the approximation,
%
\begin{equation}
\begin{split}
  &\int \frac{p(x_{k+1} \,|\, y_{1:M}) \, p(x_{k+1}\,|\,x_{k})}
           {p(x_{k+1}\,|\,y_{1:k})} \, dx_{k+1} \\
  &\approx
  \sum_i w^{(i)}_{k+1|M} \,
  \frac{p(x_{k+1}^{(i)}\,|\,x_{k})}{p(x_{k+1}^{(i)}\,|\,y_{1:k})} \\
  &\approx
  \sum_i w^{(i)}_{k+1|M} \,
  \frac{p(x_{k+1}^{(i)}\,|\,x_{k})}
   {\left[ \sum_j w_k^{(j)} \, p(x_{k+1}^{(i)}\,|\,x_{k}^{(j)}) \right]}.
\end{split}
\end{equation}
%
By substituting the SIR filter approximation and the approximation above to Equation \eqref{eq:fbsmooth} we get,
%
\begin{equation}
\begin{split}
  &p(x_k\,|\,y_{1:M}) \\
  &= p(x_k\,|\,y_{1:k})
     \int \left[ \frac{p(x_{k+1}\,|\,x_k)
                 \, p(x_{k+1}\,|\,y_{1:M})}
                 {p(x_{k+1}\,|\,y_{1:k})} \right]
         dx_{k+1} \\
  &\approx
  \sum_l
  \sum_i w^{(i)}_{k+1|M} \,
  \frac{w_k^{(l)} \,  p(x_{k+1}^{(i)}\,|\,x_{k}^{(l)})}
   {\left[ \sum_j w_k^{(j)} \, p(x_{k+1}^{(i)}\,|\,x_{k}^{(j)}) \right]} \, \delta(x_k - x_{k}^{(l)}),
\end{split}
\label{eq:rw}
\end{equation}
%
which gives the following recursion for the weights,
%
\begin{equation}
\begin{split}
  w^{(l)}_{k|M} \propto
   \sum_i w^{(i)}_{k+1|M} \,
  \frac{w_k^{(l)} p(x_{k+1}^{(i)}\,|\,x_{k}^{(l)})}
   {\left[ \sum_j w_k^{(j)} \, p(x_{k+1}^{(i)}\,|\,x_{k}^{(j)}) \right]}.
\end{split}
\end{equation}
%
Thus we get the following algorithm.
%
\begin{algo}[Reweighting particle smoother]
  \label{alg:rwsmooth} \index{Reweighting particle smoother}
  Given the weighted set of particles $\{ w_k^{(i)}, x_k^{(i)} ~|~
  i=1,\ldots,N \}$ representing the filtering distribution, we can
  form approximations to the marginal smoothing distributions as
  follows:
\begin{itemize}
\item Start by setting $w_{M|M}^{(i)} = w_M^{(i)}$ for $i=1,\ldots,N$.
\item For each $k=M-1,\ldots,0$ do the following:
  \begin{itemize}
  \item Compute new importance weights by
    \begin{equation}
      \begin{split}
        w^{(i)}_{k|M} \propto
        \sum_j w^{(j)}_{k+1|M} \,
        \frac{w_k^{(i)} p(x_{k+1}^{(j)}\,|\,x_{k}^{(i)})}
        {\left[ \sum_l w_k^{(l)} \, p(x_{k+1}^{(j)}\,|\,x_{k}^{(l)}) \right]}.
      \end{split}
    \end{equation}
  \end{itemize}
\item At each step $k$ the marginal smoothing distribution can be
  approximated as
 \begin{equation}
     p(x_k\,|\,y_{1:M})
     \approx \sum_i w_{k|M}^{(i)} \, \delta(x_k - x_k^{(i)}).
  \end{equation}
\end{itemize}
\end{algo}
%
The computational complexity of the method is $O(M\,N^2)$, that is, the same as of the backward-simulation smoother with $S = N$ simulated trajectories.


\comment{Here will be a short explanation of two filter based particle
  smoothers}


\section{Algorithms for Rao-Blackwellised Particle Smoothing}

Although the generic particle filters and smoothers can be used in almost any kind of models, the required number of samples for a sufficient accuracy can be high. The efficiency of sampling can be improved by Rao-Blackwellisation \cite{Doucet+Godsill+Andrieu:2000,Chen+Liu:2000}, where part of the state is marginalised out in closed form, and only the remaining part is sampled. Because the sampled space has a lower dimension, fewer particles are required. Such closed-form marginalisation is possible, for example, in the conditionally linear Gaussian models of Equations \eqref{eq:condgauss}.

\subsection{Rao-Blackwellised Filter/Smoother}

The Rao-Blackwellised particle filter (RBPF) \cite{Doucet+Godsill+Andrieu:2000,Chen+Liu:2000} is the algorithm which results if we marginalise the linear state $z_k$ from the conditionally linear Gaussian model \eqref{eq:condgauss}, and use SIR for the non-Gaussian part $u_k$ only. At each step the approximation to the filtering density is a mixture of Gaussians,
%
\begin{equation}
    p(u_k,z_k\,|\,y_{1:k})
    = \sum_i w_k^{(i)}
  \mathrm{N}(z_k\,|\,m_k^{(i)},P_k^{(i)}) \, \delta(u_k - u_k^{(i)}),
\nonumber
\end{equation}
%
where $u_k^{(i)}$ are the latent variable samples and $m_k^{(i)}$, $P_k^{(i)}$ are the mean and covariance of the Kalman filter conditioned on the corresponding history of latent variables. Analogously to the non-Rao-Blackwellised case, we can obtain a \cite{Kitagawa:1996} type of approximation to the smoothing solution along with the filtering by storing and resampling the histories $u^{(i)}_{0:k}$ and the corresponding Kalman filter means and covariances as well. The marginal smoothing solution for $u^{(i)}_{0:k}$ is then the following,
%
\begin{equation}
  p(u_{0:k}\,|\,y_{1:k}) \approx
  \sum_i w_k^{(i)} \delta(u_{0:k} - u_{0:k}^{(i)}).
\label{eq:rbpf_usmooth}
\end{equation}
%
The approximation to the linear part of the smoothing distribution can then be obtained by running Rauch-Tung-Striebel (RTS) smoothers for each of the histories to yield smoothed means and covariances $m^{s,(i)}_{0:n}$ and $P^{s,(i)}_{0:n}$. The algorithm is the following,
%
\begin{algo}[Rao-Blackwellised particle filter/smoother] \label{eq:rbpfs}
  Given the filtering model \eqref{eq:condgauss} and
  measurements $\{y_k\}$:
\begin{itemize}
\item Start with a weighted particle set representing the prior
  distribution: $\{ w_0^{(i)}, u_0^{(i)}, m_0^{(i)}, P_0^{(i)} \,|\,i=1,\ldots,N \}$.
\item For $k=1,\ldots,M$:
\begin{enumerate}
\item Sample from the importance distributions:
  \begin{equation}
     u_k^{(i)} \sim q(u_{k} \, | \, u_{0:k-1}^{(i)},y_{1:k})
  \end{equation}
%
  and append the samples to the latent variable histories:
%
  \begin{equation}
     u_{0:k}^{(i)} = (u_{0:k-1}^{(i)},u_k^{(i)}).
  \end{equation}

\item Do Kalman filter predictions and updates:
%
  \begin{equation}
  \begin{split}
    m_k^{-(i)} &= A(u_{k}^{(i)}) \, m_{k-1}^{(i)} \\
    P_k^{-(i)} &= A(u_{k}^{(i)}) \, P_{k-1}^{(i)} \, A^T(u_{k}^{(i)})
                + Q(u_{k}^{(i)}) \\
   \mu_k^{(i)} &= H(u_{k}^{(i)}) \, m_k^{-(i)} \\
     S_k^{(i)} &= H(u_{k}^{(i)}) \, P_{k}^{-(i)} \, H^T(u_{k}^{(i)})
                + R(u_{k}^{(i)}) \\
     K_k^{(i)} &= P_{k}^{-(i)} \, H^T(u_{k}^{(i)}) \, [S_k^{(i)}]^{-1} \\
     m_k^{(i)} &= m_k^{-(i)} + K_k^{(i)} \, [y_k - \mu_k^{(i)}] \\
     P_k^{(i)} &= P_{k}^{-(i)} - K_k^{(i)} \, S_k^{(i)} \, [K_k^{(i)}]^T.
  \end{split}
  \end{equation}
%
  and append the updated mean and covariance to the histories:
%
  \begin{equation}
  \begin{split}
     m_{0:k}^{(i)} &= (m_{0:k-1}^{(i)},m_k^{(i)}) \\
     P_{0:k}^{(i)} &= (P_{0:k-1}^{(i)},P_k^{(i)}).
  \end{split}
  \end{equation}

\item Compute importance weights as follows:
  \begin{equation}
    w_k^{(i)} \propto w_{k-1}^{(i)}
     \frac{p(u_{k}^{(i)} \, | \, u_{k-1}^{(i)}) \,
          \mathrm{N}(y_k \, | \, \mu_k^{(i)},S_k^{(i)})}
          {q(u_{k}^{(i)} \, | \, u_{0:k-1}^{(i)},y_{1:k})}. \\
  \end{equation}

\item Resample the set of histories $\{
  w_k^{(i)},u_{0:k}^{(i)},m_{0:k}^{(i)},P_{0:k}^{(i)}~:~i=1,\ldots,N
  \}$ if needed.
\end{enumerate}

\item Finally, run RTS smoothers over the histories $u_{0:M}^{(i)}, m_{0:M}^{(i)}, P_{0:M}^{(i)}$
  to get the smoothing results $m_k^{(i),s},P_k^{(i),s}$ for $k=0,\ldots,M$ and $i=1,\ldots,N$.
  \comment{Should maybe list the smoother equations here}.

\item The approximation to the marginal smoothing density at step $k$
  is then
%
\begin{equation}
  p(u_k,z_k\,|\,y_{1:M})
  = \sum_i w_k^{(i)} \mathrm{N}(z_k\,|\,m_k^{s,(i)},P_k^{s,(i)}) \,
  \delta(u_k - \tilde{u}_k^{(i)}),
\nonumber
\end{equation}
%
where $\tilde{u}_k^{(i)}$ is the $i$th component of the history
$u_{0:k}^{(i)}$.
\end{itemize}
\end{algo}
%
The Rao-Blackwellised particle smoother in this simple form also has the same disadvantage as the SIR particle smoother, that is, the smoothed estimate of $u_{k}$ can be quite degenerate if $M \gg k$. Fortunately, the smoothed estimates of the linear states $z_k$ can still be relatively good, because their degeneracy is avoided by Rao-Blackwellisation. \comment{I think this is a dubious statement. If it were true, why are we bothering? It definitely isn't true for multi-modal joint smoothing distributions (where there is more than one distinct possible path).}


\subsection{Rao-Blackwellised Two Filter Smoothers}

\comment{RB Two-filter approaches - Brier \& co.}

\subsection{Rao-Blackwellised Backward Simulation Smoothers}
%
In the Rao-Blackwellised case, we cannot do backward simulation just by drawing samples $u_k^{(i)}$ one step at a time from the filter results (as in Algorithm \ref{alg:bssmooth}), because the linear part of the state (and thus the likelihood of the measurements) depends on the entire history $u^{(i)}_{0:k}$. Furthermore, the marginalised state $u_k$ does not have the same simple Markov property as the full state,
%
\begin{equation}
  \begin{split}
    p(u_{k}\,|\,u_{k+1},y_{1:M})
    &\ne p(u_{k}\,|\,u_{k+1},y_{1:k}),
  \end{split}
  \nonumber
\end{equation}
%
and thus we need to condition on the whole future instead of just on the next time step. Rather than just $u^{(i)}_{k}$, we must calculate weights for each filtering trajectory, $u^{(i)}_{0:k}$, when it is paired with the sampled (fixed) future state sequence, $\tilde{u}_{k+1:M}$.

\subsubsection{Using Kim's Approximation}
%
The crudest way to implement a RB backward simulation smoother is simply to neglect the dependence on future observations and then use the same reasoning as that for (\ref{eq:backsimeq}) with the $x$s replaced by $u$s. This heuristic idea is called {\em Kim's approximation} \cite{Kim:1994,Barber:2006} and more formally it results from
%
\begin{equation}
\begin{split}
 &p(u_{k} \,|\, u_{k+1:n},y_{1:n}) \\
% &= \int p(u_{k} \,|\, u_{k+1},z_{k+1},y_{1:k}) \,
%         p(z_{k+1} \,|\, u_{k+1},y_{1:n}) \, dz_{k+1} \\
  &\approx p(u_{k} \,|\, u_{k+1},y_{1:k}).
\end{split}
\end{equation}
%
In other words, we smooth the non-Gaussian variables $u_k$ by ignoring the effect of Gaussian variables $z_k$ completely. We get an algorithm which amounts to replacing the transition density $p(x_{k+1}\,|\,x_{k})$ in Algorithm \ref{alg:bssmooth} with only the nonlinear transition density $p(u_{k+1}\,|\,u_{k})$. Given a trajectory of the non-Gaussian variable, the linear Gaussian part may be estimated an RTS smoother.

\begin{algo}[Kim's approximation RB backward simulation smoother]
  \label{alg:rbbssmooth}
  Given the weighted set of particles $\{ w_k^{(i)}, u_{1:k}^{(i)} ~|~
  i=1,\ldots,N,~k=1,\ldots,M \}$ representing the filtering results
  and their histories at different times:
\begin{itemize}
\item Choose $\tilde{u}_M = u_M^{(i)}$ with probability $w^{(i)}_{M}$.
\item For $k=M-1,\ldots,0$:
\begin{enumerate}
  \item Compute new weights by
    \begin{equation}
      \begin{split}
        w^{(i)}_{k|k+1} &\propto
        w_k^{(i)} \,
        p(\tilde{u}_{k+1}\,|\,u_{k}^{(i)}),
     \end{split}
    \end{equation}
    %

  \item Choose $\tilde{u}_k = u_k^{(i)}$ with probability
    $w^{(i)}_{k|k+1}$.
  \end{enumerate}
\item The smoothing solution to the linear part can be computed by running a Kalman filter and RTS smoother over the generated trajectory $\tilde{u}_{1:M}$.
\end{itemize}
%
Given $S$ iterations of the above procedure resulting in samples $\tilde{u}_{1:n}^{(j)}$ for $j=1,\ldots,S$ as well as the corresponding smoother means and covariances $m^{s,(j)}_{1:n}$, $P^{s,(j)}_{1:n}$, the smoothing distribution can be approximated as
%
\begin{equation}
  p(u_k,z_k\,|\,y_{1:n})
  = \frac{1}{S} \sum_j \mathrm{N}(z_k\,|\,m_k^{s,(j)},P_k^{s,(j)}) \,
  \delta(u_k - \tilde{u}_k^{(j)}).
\nonumber
\end{equation}
\end{algo}

%By comparing the Equations \eqref{eq:rbbacksimeq} and \eqref{eq:simplerbrw} to the Equations \eqref{eq:backsimeq} and \eqref{eq:rw}, respectively, it is easy to see that the Rao-Blackwellization only brings in a few additional Gaussian terms. It would be tempting to simply ignore those additional terms and hope that the result still is accurate enough.

\subsubsection{Sampling the linear state}
%
Improved Rao-Blackwellised smoothing algorithms were presented by \cite{Fong+Godsill+Doucet+West:2002,Lindsten+Schon:2011}. These draw samples from the smoothing distribution of the nonlinear state, $p(u_{0:M} | y_{1:M})$, by augmenting the distribution with the linear state components and sampling linear and nonlinear states alternately.

By introducing the linear state sampling, we can exploit the Markov property in the same manner as the non-Rao-Blackwellised smoother. Supposing a future state trajectory, $\{ \tilde{u}_{k+1:M}, \tilde{z}_{k+1:M} \}$ has already been sampled. We would then like to sample,
%
\begin{equation}
\begin{split}
  &p(u_{0:k}, z_k\,|\,\tilde{u}_{k+1:M}, \tilde{z}_{k+1:M},y_{1:M}) \\
  &= p(u_{0:k}, z_k\,|\,\tilde{u}_{k+1}, \tilde{z}_{k+1},y_{1:k}) \\
  &= p(u_{0:k}\,|\,\tilde{u}_{k+1},\tilde{z}_{k+1},y_{1:k}) p(z_k\,|\, u_{0:k}, \tilde{u}_{k+1}, \tilde{z}_{k+1},y_{1:k}) \\
\end{split}
\label{eq:linsampfact}
\end{equation}

Considering the nonlinear sampling first,
%
\begin{equation}
\begin{split}
  &p(u_{0:k}\,|\,\tilde{u}_{k+1},\tilde{z}_{k+1},y_{1:k}) \\
  &= \frac{ p(\tilde{u}_{k+1}, \tilde{z}_{k+1} | u_{0:k}, y_{1:k}) p(u_{0:k}|y_{1:k}) }{ p(\tilde{u}_{k+1}, \tilde{z}_{k+1} | y_{1:k}) } \\
  &\propto p(\tilde{z}_{k+1} | u_{0:k}, \tilde{u}_{k+1}, y_{1:k}) p(\tilde{u}_{k+1} | u_k) p(u_{0:k}|y_{1:k})     .
\end{split}
\end{equation}

The linear density term is given by a Kalman filter prediction step using the moments from the forward filtering stage,
%
\begin{equation}
\begin{split}
  p(\tilde{z}_{k+1} | u_{1:k}, \tilde{u}_{k+1}, y_{1:k}) = \mathrm{N}(\tilde{z}_{k+1}\,|\, m^{-}_{k+1}, P^{-}_{k+1}), 
\end{split}
\end{equation}
%
where,
%
\begin{equation}
\begin{split}
 m^{-}_{k+1} &= A(\tilde{u}_{k+1}) \, m_{k} \\
 P^{-}_{k+1} &= A(\tilde{u}_{k+1}) \, P_{k} A(\tilde{u}_{k+1})^T + Q(\tilde{u}_{k+1})     .
\end{split}
\end{equation}
%
Substituting this, and the RBPF approximation for the nonlinear filtering density gives us,
%
\begin{equation}
\begin{split}
  &p(u_{0:k}\,|\,\tilde{u}_{k+1:M},z_{k+1},y_{1:M}) \\
  &\approx \sum_i w^{(i)}_{k|k+1} \delta(u_{0:k} - u_{0:k}^{(i)}) \\
  &w^{(i)}_{k|k+1} \propto w_k^{(i)} p(\tilde{u}_{k+1} | u_k^{(i)}) \mathrm{N}(\tilde{z}_{k+1}\,|\, m^{-(i)}_{k+1}, P^{-(i)}_{k+1})     .
\end{split}
\end{equation}

Hence, we can sample from this distribution by selecting $u_{0:k}^{(i)}$ with probability $\propto w^{(i)}_{k|k+1}$.

In the second stage, we draw a sample from the linear factor in (\ref{eq:linsampfact}). This distribution may be expanded as,
%
\begin{equation}
\begin{split}
  &p(z_k\,|\, u_{0:k}, \tilde{u}_{k+1}, \tilde{z}_{k+1},y_{1:k}) \\
  &\propto p(\tilde{z}_{k+1} | z_k, \tilde{u}_{k+1}) p(z_k\,|\, u_{0:k}, y_{1:k})
\end{split}
\end{equation}

This will be a Gaussian and its form is akin to a Kalman filter update,
%
\begin{equation}
\begin{split}
  &p(z_k\,|\, u_{0:k}, \tilde{u}_{k+1}, \tilde{z}_{k+1},y_{1:k}) \\
  &= \mathrm{N}( z_k | \tilde{m}_k , \tilde{P}_k )
\end{split}
\end{equation}
%
\begin{equation}
\begin{split}
  T_{k+1}^{(i)} &= ( A(\tilde{u}_{k+1}) P_k^{(i)} A(\tilde{u}_{k+1})^T + Q(\tilde{u}_{k+1}) )^{-1} \\
  \tilde{m}_k &= m_k^{(i)} + P_k^{(i)} A(\tilde{u}_{k+1})^T T_{k+1}^{(i)} (\tilde{z}_{k+1} - A(\tilde{u}_{k+1}) m_k^{(i)} ) \\
  \tilde{P}_k &= P_k^{(i)} - P_k^{(i)} A(\tilde{u}_{k+1})^T T_{k+1}^{(i)} A(\tilde{u}_{k+1}) P_k^{(i)}
\end{split}
\label{eq:linsamplinupd}
\end{equation}

Finally, we discard the historical nonlinear states, $u_{0:k-1}$, and proceed to the next step. Once complete, a joint sample has been generated from the full state smoothing distribution. To obtain a true Rao-Blackwellised estimate, we can now discard the linear state samples and run an RTS smoother for each nonlinear state trajectory.

The final algorithm is as follows,
%
\begin{algo}[Linear sampling RB backward simulation smoother]
  \label{alg:rbbssmooth}
  Given the weighted set of particles $\{ w_k^{(i)}, u_{1:k}^{(i)} ~|~ i=1,\ldots,N,~k=1,\ldots,M \}$ representing the filtering results and their histories at different times, and the accompanying Gaussian moments $\{ m_k^{(i)}, P_k^{(i)} ~|~ i=1,\ldots,N,~k=1,\ldots,M \}$:
\begin{itemize}
\item Choose $\tilde{u}_n = u_M^{(i)}$ with probability $w^{(i)}_{M}$.
\item For $k=M-1,\ldots,0$:
\begin{enumerate}
  \item Compute new weights by
    \begin{equation}
      \begin{split}
        w^{(i)}_{k|k+1} \propto w_k^{(i)} p(\tilde{u}_{k+1} | u_k^{(i)}) \mathrm{N}(\tilde{z}_{k+1}\,|\, m^{-(i)}_{k+1}, P^{-(i)}_{k+1})
     \end{split}
    \end{equation}
    %
    where $m^{-(i)}_{k+1}$ and $P^{-(i)}_{k+1}$ are the forward Kalman filter mean and covariance for the trajectory $u_{1:k}^{(i)}$ predicted to step $k+1$ using $\tilde{u}_{k+1}$.
  \item Choose $\tilde{u}_{k} = u_{k}^{(i)}$ with probability
    $w^{(i)}_{k|k+1}$.
  \item Samples $z_k \sim \mathrm{N}( z_k | \tilde{m}_k , \tilde{P}_k )$ where $\tilde{m}_k$ and $\tilde{P}_k$ are given by (\ref{eq:linsamplinupd}).
  \end{enumerate}
\item The smoothing solution to the linear part can be computed by discarding the sampled trajectory $\tilde{z}_{1:M}$ and running a Kalman filter and RTS smoother over the generated trajectory $\tilde{u}_{1:M}$.
\end{itemize}
%
Given $S$ iterations of the above procedure resulting in samples $\tilde{u}_{1:M}^{(j)}$ for $j=1,\ldots,S$ as well as the corresponding smoother means and covariances $m^{s,(j)}_{1:n}$, $P^{s,(j)}_{1:n}$, the smoothing distribution can be approximated as,
%
\begin{equation}
  p(u_k,z_k\,|\,y_{1:n})
  = \frac{1}{S} \sum_j \mathrm{N}(z_k\,|\,m_k^{s,(j)},P_k^{s,(j)}) \,
  \delta(u_k - \tilde{u}_k^{(j)}).
\nonumber
\end{equation}
\end{algo}

Note that this algorithm differs slightly from those described in \cite{Fong+Godsill+Doucet+West:2002,Lindsten+Schon:2011}, which instead use different, and approximate, results for the linear sampling term, $p(z_k\,|\, u_{1:k}, \tilde{u}_{k+1}, \tilde{z}_{k+1},y_{1:k})$. \comment{Is this ok? As I understand it, they both seem to ignore the future sampled states and sample from some sort of RTS approximation.}



\subsubsection{New Fully Rao-Blackwellised Smoother}

A smoother which samples the linear states is not truly Rao-Blackwellised. Rather, an algorithm is desirable which samples only the nonlinear states, with the linear part only appearing in analytical weight calculations. Such an algorithm is presented in this section.

In order to do backward sampling for $u_{1:M}$, we will need an efficient way to evaluate the following distribution for different values of $u_{0:k}$,
%
\begin{equation}
  p(u_{0:k}\,|\,u_{k+1:M},y_{1:M})
  = \frac{p(u_{0:M}\,|\,y_{1:M})}{p(u_{k+1:M}\,|\,y_{1:M})}.
\label{eq:norm_bp}
\end{equation}
%
Because we are sampling a single trajectory, we can consider the denominator in \eqref{eq:norm_bp} as fixed. Thus we are interested in
evaluating,
%
\begin{equation}
  p(u_{0:M}\,|\,y_{1:M}) \propto p(y_{1:M}\,|\,u_{0:M}) \, p(u_{0:M}).
\end{equation}
%
The prior $p(u_{0:M}) = p(u_0) \, \prod_{j=1}^M p(u_{j}\,|\,u_{j-1})$ is easy to evaluate; the problematic term is the marginal likelihood $p(y_{1:M}\,|\,u_{0:M})$. It could be calculated in a brute force way, by running a Kalman filter through the data.  However, at each sampling time $k$ we then would need to run the Kalman filters in each particle $i$ over the entire sampled future $\tilde{u}_{k+1:M}$. Thus we need a more efficient approach.

We can split up the dependency between the past and future by augmenting the distribution with $z_k$, to exploit the Markovian property of the dynamics,
%
\begin{equation}
\begin{split}
  & p(u_{0:M}\,|\,y_{1:M}) = \int p(u_{0:M}, z_k \,|\,y_{1:M}) dz_k \\
  &= \int \frac{ p(y_{k+1:M} | u_{0:M}, z_k) p(u_{0:M}, z_k | y_{1:k}) }{ p(y_{k+1:M} | y_{1:k}) } dz_k \\
  &\propto p(u_{0:k} | y_{1:k}) p(u_{k+1} | u_{k}) \int p(y_{k+1:M} | u_{0:M}, z_k) p(z_k | u_{0:k}, y_{1:k}) dz_k    ,
\end{split}
\label{eq:split}
\end{equation}

where, again, we have ignored terms which do not depend on $u_{0:k}$.


%
%We can split up the dependency between the past and future in the
%marginal likelihood by augmenting with $z_k$, which gives:
%%
%\begin{equation}
%\begin{split}
%  &p(y_{1:M},z_k\,|\,u_{0:M}) \\
%%  &=
%%    p(y_{1:M}\,|\,z_k,u_{0:M}) \, p(z_k\,|\,u_{0:M}) \\
%%  &=
%%    p(y_{k+1:M}\,|\,z_k,u_{0:M},y_{1:k}) \,
%%    p(y_{1:k}\,|\,z_k,u_{0:M}) \,
%%    p(z_k\,|\,u_{0:M}) \\
%%  &=
%%    p(y_{k+1:M}\,|\,z_k,u_{k:M}) \,
%%    p(y_{1:k}\,|\,z_k,u_{0:k}) \,
%%    p(z_k\,|\,u_{0:M}) \\
%%  &=
%%    p(y_{k+1:M}\,|\,z_k,u_{k:M}) \,
%%    \frac{p(z_k \,|\,y_{1:k},u_{0:k}) \, p(y_{1:k}\,|\,u_{0:k})}
%%         {p(z_k \,|\,u_{0:k})}
%%    \, p(z_k\,|\,u_{0:M}) \\
%%  &=
%%    p(y_{k+1:M}\,|\,z_k,u_{k:M}) \,
%%    p(z_k \,|\,y_{1:k},u_{0:k}) \,
%%    p(y_{1:k}\,|\,u_{0:k}) \\
%  &=
%    p(y_{k+1:M}\,|\,z_k,u_{0:M},y_{1:k}) \,
%    p(y_{1:k},z_k \,|\, u_{0:M}) \\
%  &=
%    p(y_{k+1:M}\,|\,z_k,u_{k:M}) \,
%    p(z_k \,|\, y_{1:k},u_{0:M}) \,
%    p(y_{1:k} \,|\, u_{0:M}) \\
%  &=
%    p(y_{k+1:M}\,|\,z_k,u_{k:M}) \,
%    p(z_k \,|\, y_{1:k},u_{0:k}) \,
%    p(y_{1:k} \,|\, u_{0:k}).
%\end{split}
%\label{eq:split}
%\end{equation}
%
%The last two terms in Equation \eqref{eq:split} are given by:
%%
%\begin{equation}
%\begin{split}
%  p(z_k \,|\,y_{1:k},u_{0:k})
%  &= \mathrm{N}(z_k\,|\,m_k,P_k) \\
%  p(y_{1:k}\,|\,u_{0:k})
%  &= \prod_{j=1}^k \mathrm{N}(y_j\,|\,H(u_j) \, m^-_j,S_j),
%\end{split}
%\end{equation}
%%
%where the means and covariance can be computed
%with the Kalman filter recursions
%%
%\begin{equation}
%\begin{split}
% m^-_k &= A(u_{k-1}) \, m_{k-1} \\
% P^-_k &= A(u_{k-1}) \, P_{k-1} A^T(u_{k-1}) + Q(u_{k-1}) \\
%   S_k &= H(u_{k}) \, P^-_k \, H^T(u_{k}) + R(u_{k}) \\
%   K_k &= P^-_k \, H^T(u_{k}) \, S_k^{-1} \\
%   m_k &= m^-_k + K_k \, [y_k - H(u_{k}) \, m^-_k] \\
%   P_k &= P^-_k - K_k \, S_k \, K_k^T.
%\end{split}
%\end{equation}
The last term in Equation \eqref{eq:split} is given by,
%
\begin{equation}
\begin{split}
  p(z_k \,|\, u_{0:k}, y_{1:k})
  &= \mathrm{N}(z_k\,|\,m_k,P_k) \\
  p(y_{1:k}\,|\,u_{0:k})
  &= \prod_{j=1}^k \mathrm{N}(y_j\,|\,H(u_j) \, m^-_j,S_j),
\end{split}
\end{equation}
%
where the means and covariance can be computed with the Kalman filter recursions,
%
\begin{equation}
\begin{split}
 m^-_k &= A(u_{k}) \, m_{k-1} \\
 P^-_k &= A(u_{k}) \, P_{k-1} A^T(u_{k}) + Q(u_{k}) \\
   S_k &= H(u_{k}) \, P^-_k \, H^T(u_{k}) + R(u_{k}) \\
   K_k &= P^-_k \, H^T(u_{k}) \, S_k^{-1} \\
   m_k &= m^-_k + K_k \, [y_k - H(u_{k}) \, m^-_k] \\
   P_k &= P^-_k - K_k \, S_k \, K_k^T.
\end{split}
\end{equation}
%
The first term in the integral in Equation \eqref{eq:split} can be computed with a backward Kalman filter. First assume that there exists a mean and covariance $m^{b}_{k+1},P^{b}_{k+1}$ and a normalisation constant $Z_{k+1}$ such that,
%
\begin{equation}
\begin{split}
  p(y_{k+1:M} | z_{k+1},u_{k+1:M})
  = Z_{k+1} \mathrm{N}(z_{k+1} | m^{b}_{k+1}, P^{b}_{k+1})     .
\end{split}
\label{eq:normconst}
\end{equation}
%
By the two-filter smoother equations we then get,
%
\begin{equation}
\begin{split}
  &p(y_{k+1:M} \,|\, z_{k},u_{k+1:M}) \\
  &=
  \int
  p(y_{k+1:M} \,|\, z_{k+1},u_{k+1:M}) \,
  p(z_{k+1} \,|\, z_{k},u_{k+1}) \, dz_{k+1} \\
  &= \int
  Z_{k+1} \mathrm{N}(z_{k+1} \,|\, m^{b}_{k+1}, P^{b}_{k+1}) \\
  &\qquad \times \mathrm{N}(z_{k+1} \,|\,A(u_{k+1}) \, z_k, Q(u_{k+1})) \, dz_{k+1} \\
  &=
  Z_{k+1} |\det A(u_{k+1})|^{-1}
  \mathrm{N}(z_{k} \,|\, m^{-b}_{k}, P^{-b}_{k}),
\end{split}
\end{equation}
%
where,
%
\begin{equation}
\begin{split}
   m^{-b}_k &= A^{-1}(u_{k+1}) \, m^b_{k+1} \\
   P^{-b}_k &= A^{-1}(u_{k+1}) \, (Q(u_{k+1}) + P^b_{k+1}) \, A^{-T}(u_{k+1}),
\end{split}
\label{eq:bkf_predict}
\end{equation}
%
and,
%
\begin{equation}
\begin{split}
  &p(y_{k:M} \,|\, z_{k},u_{k:M}) \\
  &= p(y_{k} \,|\, z_{k},u_{k:M}) \, p(y_{k+1:M} \,|\, z_{k},u_{k+1:M}) \\
  &= \mathrm{N}(y_{k} \,|\, H(u_k) \, z_k, R(u_k)) \\
  &\quad \times Z_{k+1} |\det A(u_{k+1})|^{-1}
  \mathrm{N}(z_{k} \,|\, m^{-b}_{k}, P^{-b}_{k}) \\
  &= Z_{k+1} |\det A(u_{k+1})|^{-1} \, \mathrm{N}(y_{k} \,|\, \mu^b_k, S^b_k) \,
     \mathrm{N}(z_{k} \,|\, m^{b}_{k}, P^{b}_{k}) \\
  &= Z_{k} \,
     \mathrm{N}(z_{k} \,|\, m^{b}_{k}, P^{b}_{k}),
\end{split}
\end{equation}
%
where,
%
\begin{equation}
\begin{split}
    \mu^b_k &= H(u_k) \, m^{-b}_k \\
      S^b_k &= H(u_k) P^{-b}_k \, H^T(u_k) + R(u_k) \\
      K^b_k &= P^{-b}_k \, H^T(u_k) \, (S^b_k)^{-1} \\
      m^b_k &= m^{-b}_k + K^b_k \, [y_k - \mu^b_k] \\
      P^b_k &= P^{-b}_k - K^b_k \, S^b_k \, (K^b_k)^T \\
        Z_k &= Z_{k+1} \, |\det A(u_{k+1})|^{-1} \,
               \mathrm{N}(y_k | \mu^b_k,S^b_k).
\end{split}
\end{equation}
%
Finally, we recognise the first term in \eqref{eq:split} as the nonlinear state filtering distribution estimated by the RBPF. Substituting this, and the Gaussian terms gives,
%





%Substituting the results into Equation \eqref{eq:split} gives
%%
%\begin{equation}
%\begin{split}
%  &p(y_{1:M},z_k\,|\,u_{0:M}) \\
%  &= Z_{k+1} \, |\det A(u_k)|^{-1} \,
%  \mathrm{N}(z_{k} \,|\, m^{-b}_{k}, P^{-b}_{k}) \\
%  &\qquad \times
%  \mathrm{N}(z_k\,|\,m_k,P_k) \,
%  \prod_{j=1}^k \mathrm{N}(y_j\,|\,H(u_j) \, m^-_j,S_j),
%\end{split}
%\end{equation}
%%
%Integrating over $z_k$ then gives
%%Substituting the results into Equation \eqref{eq:split} and
%%integrating over $z_k$ then gives
%%
%\begin{equation}
%\begin{split}
%  &p(y_{1:M}\,|\,u_{0:M}) \\
%  &= Z_{k+1} \, |\det A(u_k)|^{-1}
%  \mathrm{N}(m_k\,|\,m^{-b}_{k},P_k+P^{-b}_{k}) \\
%  &\qquad \times
%  \prod_{j=1}^k \mathrm{N}(y_j\,|\,H(u_j) \, m^-_j,S_j).
%\end{split}
%\end{equation}
%%
%By combining with the prior of $u_{0:M}$ and ignoring the
%terms that only depend on $u_{k+1:M}$, we get
%%
%\begin{equation}
%\begin{split}
%  &p(u_{0:k}\,|\,u_{k+1:M},y_{1:M}) \\
%  &\propto
%  |\det A(u_k)|^{-1}
%  \mathrm{N}(m_k\,|\,m^{-b}_{k},P_k+P^{-b}_{k}) \, p(u_{k+1}\,|\,u_{k}) \\
%  &\quad \times
%  p(u_0) \,
%  \prod_{j=1}^k p(u_{j}\,|\,u_{j-1}) \,
%  \mathrm{N}(y_j\,|\,H(u_j) \, m^-_j,S_j).
%\end{split}
%\end{equation}
%%
%The term on the second line is just the distribution approximated by
%the RBPF, (Eq. \eqref{eq:rbpf_usmooth}). Substituting the particle
%approximation then gives



%
%
\begin{equation}
\begin{split}
  &p(u_{0:k}\,|\,u_{k+1:M},y_{1:M}) = \sum_i w_{k|k+1}^{(i)} \delta(u_{0:k} - u^{(i)}_{0:k}) \\
  &w_{k|k+1}^{(i)} \propto w^{(i)}_k |\det A(u_{k+1})|^{-1} \mathrm{N}(m_k^{(i)}\,|\,m^{-b}_{k},P_k^{(i)}+P^{-b}_{k}) \, p(u_{k+1}\,|\,u_{k})
\end{split}
\label{eq:rbbacksimeq}
\end{equation}
%
The final algorithm is the following.
%
\begin{algo}[Rao-Blackwellised backward simulation smoother]
  \label{alg:rbbssmooth}
  Given the weighted set of particles $\{ w_k^{(i)}, u_{1:k}^{(i)} ~|~ i=1,\ldots,N,~k=1,\ldots,M \}$ representing the filtering results and their histories at different times, and the accompanying Gaussian moments $\{ m_k^{(i)}, P_k^{(i)} ~|~ i=1,\ldots,N,~k=1,\ldots,M \}$:
\begin{itemize}
\item Choose $\tilde{u}_M = u_M^{(i)}$ with probability $w^{(i)}_{M}$. Also initialise $m_M^b$ and $P_M^b$. See text.
\item For $k=M-1,\ldots,0$:
\begin{enumerate}
  \item Compute new weights by
    \begin{equation}
      \begin{split}
        w_{k|k+1}^{(i)} \propto w^{(i)}_k |\det A(u_{k+1})|^{-1} \mathrm{N}(m_k^{(i)}\,|\,m^{-b}_{k},P_k^{(i)}+P^{-b}_{k}) \, p(u_{k+1}\,|\,u_{k}^{(i)})
     \end{split}
    \end{equation}
    %
    where $m^{(i)}_k$ and $P^{(i)}_k$ are the forward Kalman filter mean and covariance for the trajectory $u_{1:k}^{(i)}$, and     $m^{-b,(i)}_{k}$ and $P^{-b,(i)}_{k}$ are the backward filter  results backward predicted to step $k$ using $\tilde{u}_{k+1}^{(i)}$. \item Choose $\tilde{u}_k = u_k^{(i)}$ with probability $w^{(i)}_{k|k+1}$.
  \end{enumerate}
\item The smoothing solution to the linear part can be computed by running a Kalman filter and RTS smoother over the generated trajectory $\tilde{u}_{0:M}$. In practice, we can also compute the solution during the backward simulation by combining the computed forward and backward filtering solutions using the two-filter smoother formulas.
\end{itemize}
%
Given $S$ iterations of the above procedure resulting in samples $\tilde{u}_{0:M}^{(j)}$ for $j=1,\ldots,S$ as well as the corresponding smoother means and covariances $m^{s,(j)}_{1:n}$, $P^{s,(j)}_{1:n}$, the smoothing distribution can be approximated as,
%
\begin{equation}
  p(u_k,z_k\,|\,y_{1:n})
  = \frac{1}{S} \sum_j \mathrm{N}(z_k\,|\,m_k^{s,(j)},P_k^{s,(j)}) \,
  \delta(u_k - \tilde{u}_k^{(j)}).
\nonumber
\end{equation}
\end{algo}


\subsection{Rao-Blackwellised Reweighting Smoothers}
%
A Rao-Blackwellised version of the reweighting particle smoother
in Algorithm \ref{alg:rwsmooth} can be derived by considering
the optimal smoothing equation:
%
\begin{equation}
\begin{split}
 &p(z_{k},u_{k} \,|\, y_{1:n}) \\
 &= p(z_{k},u_{k}\,|\,y_{1:k}) \\
 &\times \iint
    \frac{p(z_{k+1},u_{k+1} \,|\, y_{1:n}) \, p(z_{k+1},u_{k+1}\,|\,z_{k},u_{k})}
         {p(z_{k+1},u_{k+1}\,|\,y_{1:k})} \, dz_{k+1} \, du_{k+1} \\
\end{split}
\end{equation}
%
From the RBPF results we get the following approximations,
%
\begin{equation}
\begin{split}
  &p(z_k,u_{k} \,|\, y_{1:k}) \\
  &\approx \sum_j w_k^{(j)} \, \mathrm{N}(z_{k} | m^{(j)}_{k}, P^{(j)}_{k}) \,
    \delta(u_{k} - u_{k}^{(j)}) \\
  &p(z_{k+1},u_{k+1} \,|\, y_{1:k}) \\
  &\approx \sum_j w_k^{(j)} \, p(u_{k+1}\,|\,u_k^{(j)}) \, \mathrm{N}(z_{k+1} | m^{-(j)}_{k+1}, P^{-(j)}_{k+1}),
\end{split}
\end{equation}
%
and the dynamic model is,
%
\begin{equation}
\begin{split}
  &p(z_{k+1},u_{k+1} \,|\, z_{k},u_{k}) \\
  &= \mathrm{N}(z_{k+1} | A(u_{k+1}) \, z_k, Q(u_{k+1})) \, p(u_{k+1} | u_k).
\end{split}
\end{equation}
%
Assume that the smoothing distribution of the next step is a mixture
of Gaussians as follows,
%
\begin{equation}
\begin{split}
  &p(u_{k+1},z_{k+1}\,|\,y_{1:n}) \\
  &=
    \sum_{i} w^{(i)}_{k+1|n} \, \mathrm{N}(z_{k+1}\,|\,\tilde{m}_{k+1}^{(i),s},\tilde{P}_{k+1}^{(i),s}) \,
    \delta(u_{k+1} - u_{k+1}^{(i)}),
\end{split}
\end{equation}
%
where $u_{k+1}^{(i)}$ are the samples from the RBPF at time step $k+1$, and $\tilde{m}_{k+1}^{(i),s}$ and $\tilde{P}_{k+1}^{(i),s}$ are the means and covariances of the mixture components conditioned on those trajectories.
%
%Note that the marginal distribution of $u_{k+1}$ is then given as
%
%\begin{equation}
%\begin{split}
%  p(u_{k+1}\,|\,y_{1:n})
%  &=
%    \int \sum_{i} w^{(i)}_{k+1|n} \, \mathrm{N}(z_{k+1}\,|\,\tilde{m}_{k+1}^{(i),s},\tilde{P}_{k+1}^{(i),%s}) \,
%    \delta(u_{k+1} - u_{k+1}^{(i)}) \, dz_{k+1} \\
%  &=
%  \sum_{i} w^{(i)}_{k+1|n} \, \delta(u_{k+1} - u_{k+1}^{(i)}).
%\end{split}
%\end{equation}
%
Substituting above into the smoothing equation gives
%
\begin{equation}
\begin{split}
 &p(z_{k},u_{k} \,|\, y_{1:n}) \\
% &= \sum_j w_k^{(j)} \, \mathrm{N}(z_{k} | m^{(j)}_{k}, P^{(j)}_{k}) \, \delta(u_{k} - u_{k}^{(j)}) \\
% &\times \iint
%    \frac{\sum_{i} w^{(i)}_{k+1|n} \,
%          \mathrm{N}(z_{k+1}\,|\,\tilde{m}_{k+1}^{(i),s},\tilde{P}_{k+1}^{(i),s}) \,
%          \delta(u_{k+1} - u_{k+1}^{(i)}) \,
%         \mathrm{N}(z_{k+1} | A(u_k) \, z_k, Q(u_k)) \, p(u_{k+1} | u_k)}
%         {\sum_l w_k^{(l)} \, p(u_{k+1}\,|\,u_k^{(l)}) \, \mathrm{N}(z_{k+1} | m^{-(l)}_{k+1}, P^{-(l)}_{k+1})}
%         \, dz_{k+1} \, du_{k+1} \\
% &= \sum_{ij} w_k^{(j)} \, \mathrm{N}(z_{k} | m^{(j)}_{k}, P^{(j)}_{k}) \, \delta(u_{k} - u_{k}^{(j)}) \\
% &\times \int
%    \frac{w^{(i)}_{k+1|n} \,
%          \mathrm{N}(z_{k+1}\,|\,\tilde{m}_{k+1}^{(i),s},\tilde{P}_{k+1}^{(i),s}) \,
%          \mathrm{N}(z_{k+1} | A(u_k) \, z_k, Q(u_k)) \, p(u_{k+1}^{(i)} | u_k)}
%        {\sum_l w_k^{(l)} \, p(u_{k+1}^{(i)}\,|\,u_k^{(l)}) \,
%          \mathrm{N}(z_{k+1} | m^{-(l)}_{k+1}, P^{-(l)}_{k+1})} \, dz_{k+1} \\
% &= \sum_{ij} w^{(i)}_{k+1|n} \, w_k^{(j)} \,
%      p(u_{k+1}^{(i)} | u_k) \, \delta(u_{k} - u_{k}^{(j)}) \\
% &\times \int
%    \frac{\mathrm{N}(z_{k+1}\,|\,\tilde{m}_{k+1}^{(i),s},\tilde{P}_{k+1}^{(i),s}) \,
%          \mathrm{N}(z_{k+1} | A(u_k) \, z_k, Q(u_k))}
%        {\sum_l w_k^{(l)} \, p(u_{k+1}^{(i)}\,|\,u_k^{(l)}) \,
%          \mathrm{N}(z_{k+1} | m^{-(l)}_{k+1}, P^{-(l)}_{k+1})} \, dz_{k+1} \\
%  &\times \mathrm{N}(z_{k} | m^{(j)}_{k}, P^{(j)}_{k}).
 &= \sum_{ij} w^{(i)}_{k+1|n} \, w_k^{(j)} \,
      p(u_{k+1}^{(i)} | u_k^{(j)}) \, \delta(u_{k} - u_{k}^{(j)}) \\
 &\times \int
    \frac{\mathrm{N}(z_{k+1}\,|\,\tilde{m}_{k+1}^{(i),s},\tilde{P}_{k+1}^{(i),s}) \,
          \mathrm{N}(z_{k+1} | A(u_k^{(j)}) \, z_k, Q(u_k^{(j)}))}
        {\sum_l w_k^{(l)} \, p(u_{k+1}^{(i)}\,|\,u_k^{(l)}) \,
          \mathrm{N}(z_{k+1} | m^{-(l)}_{k+1}, P^{-(l)}_{k+1})} \, dz_{k+1} \\
  &\times \mathrm{N}(z_{k} | m^{(j)}_{k}, P^{(j)}_{k}).
\end{split}
\end{equation}
%
We can now form approximation to the backward marginal smoothing (reweighting smoothing) by using moment matching analogously to the Interacting Multiple Models (IMM) algorithm \cite{Bar-Shalom+Li+Kirubarajan:2001} (cf. \cite{Barber:2006}). For each $j$ we can compute the normalisation constant, mean and covariance and form a mixture Gaussian approximation to the above distribution. We can use, for example, Gauss--Hermite or cubature integration. Another way would be to use Monte Carlo integration. \comment{This needs further explanation.}

\comment{Algorithm about here}

Another simple way is to approximate the denominator using a Gaussian distribution. We can, for example match the moments to obtain approximation of the form,
%
\begin{equation}
\begin{split}
  &\sum_l w_k^{(l)} \, p(u_{k+1}^{(i)}\,|\,u_k^{(l)}) \,
            \mathrm{N}(z_{k+1} | m^{-(l)}_{k+1}, P^{-(l)}_{k+1}) \\
  &\approx
  C_i \, \mathrm{N}(z_{k+1} | \hat{m}^{-(i)}_{k+1}, \hat{P}^{-(i)}_{k+1})
\end{split}
\end{equation}
%
where,
%
\begin{equation}
  C_i = \sum_l w_k^{(l)} \, p(u_{k+1}^{(i)}\,|\,u_k^{(l)}).
\end{equation}
\comment{What about $\hat{m}^{-(i)}_{k+1}, \hat{P}^{-(i)}_{k+1}$? Are they from moment matching? Formulas? Where does the $C_i$ formula come from?}
%
Then we get,
%
\begin{equation}
\begin{split}
 &p(z_{k},u_{k} \,|\, y_{1:n}) \\
 &= \sum_{ij} w^{(i)}_{k+1|n} \, w_k^{(j)} \,
      p(u_{k+1}^{(i)} | u_k^{(j)}) \, \delta(u_{k} - u_{k}^{(j)}) \\
 &\times \int
    \frac{\mathrm{N}(z_{k+1}\,|\,\tilde{m}_{k+1}^{(i),s},\tilde{P}_{k+1}^{(i),s}) \,
          \mathrm{N}(z_{k+1} | A(u_k^{(j)}) \, z_k, Q(u_k^{(j)}))}
        {C_i \, \mathrm{N}(z_{k+1} | \hat{m}^{-(i)}_{k+1}, \hat{P}^{-(i)}_{k+1})} \, dz_{k+1} \\
 &\times \mathrm{N}(z_{k} | m^{(j)}_{k}, P^{(j)}_{k}) \\
 &= \sum_{ij} w^{(i)}_{k+1|n} \,
    \frac{w_k^{(j)} \, p(u_{k+1}^{(i)} | u_k^{(j)})}
         {\sum_l w_k^{(l)} \, p(u_{k+1}^{(i)}\,|\,u_k^{(l)})} \, \delta(u_{k} - u_{k}^{(j)}) \\
 &\times \mathrm{N}(z_{k+1}\,|\,\tilde{m}_{k}^{(j,i),s},\tilde{P}_{k}^{(j,i),s}) \,
\end{split}
\label{eq:simplerbrw}
\end{equation}
%
\comment{Is there another approximation here in the last step? The linear terms don't look quite like an RTS smoother - wrong denominator?}
where the means $\tilde{m}_{k}^{(j,i),s}$ and covariances $\tilde{P}_{k}^{(j,i),s}$ are computed by RTS equations. We can then reduce this into a Gaussian mixture with $N$ terms by a simple moment matching procedure.

\comment{Algorithm about here}

Note that while Kim's approximation could be used to formulate a reweighting-type smoother, we would be left with the awkward problem of how to estimate the linear state marginal smoothing distributions, $p(z_k | u_k, y_{1:M})$. As a result, this option has not been pursued.

%\subsubsection{Using Kim's Approximation}
%%
%
%
%In the reweighting smoother case Kim's approximation can indeed by
%used for compute the weights for the representations
%%
%\begin{equation}
%\begin{split}
%  p(u_{k}\,|\,y_{1:n})
%  &\approx \sum_{i} w^{(i)}_{k|n} \, \delta(u_{k} - u_{k}^{(i)}),
%\end{split}
%\end{equation}
%%
%but for the linear part of the state we need additional
%approximations. As the approximations require quite much the same
%computations as the approximations presented in the previous section,
%their applicability seems to be quite limited.


\section{Computational Considerations}

\subsection{Initialization of Backward Simulation Recursion}
%
The recursion in Section XXX is based on the assumption that the normalisation constant in the Equation \eqref{eq:normconst} exists and is finite, making the ``inversion'' of the Gaussian density possible.  Unfortunately, at the last step, when $k+1 = M$ it is often the case that such a normalisation constant does not exist. This issue is related to the well-known normalisation problem of two-filter smoothers. In the case of linear smoothing this problem can be overcome by using an information filter and defining the distribution at the last step as being formally singular. Unfortunately, although we only need to use a linear smoother, this approach does not solve the problem of undefined normalisation constant in our case.

Fortunately, if the system is observable, the normalisation constant is guaranteed to become finite at some point. One simple, approximate way to initialise the filter is to start by using the Kim's approximation based smoother until the normalisation constant becomes finite (say $s$ steps). An alternative, exact method is to calculate the first term of Equation \eqref{eq:split} directly, using the observation density with an augmented vector of all the observations from $k+1$ to $M$. The form for this augmented observation density is given by \cite{Kitagawa:1994}. Thus, it is not necessary to invert the Gaussian to calculate the sampling weights until the normalisation constant becomes finite.

If the observation matrix, $H(u_M)$, is full rank, then the backwards filter may initialised by taking the final observation likelihood density and ``inverting'' it,
%
\begin{equation}
\begin{split}
  p(y_M | z_M, u_M) &= \mathrm{N}( y_M | H(u_M) z_M , R(u_M) ) \\
  &= Z_M \mathrm{N}( z_M | m_M^b , P_M^b )
\end{split}
\end{equation}
%
where,
%
\begin{equation}
\begin{split}
  m_M^b &= H(u_M)^{-1} y_M \\
  P_M^b &= H(u_M)^{-1} R(u_M) H(u_M)^{-T} \\
  Z_M   &= |\det H(u_M)|     .
\end{split}
\end{equation}

If $H(u_M)$ is not invertible, then this is not possible. Instead, we must wait a few steps and then invert the joint observation density. In the following, we denote $H_k = H(u_k)$, etc. for clarity.

\begin{equation}
\begin{split}
 p(y_{M-s:M}|z_{M-s}) &= \mathrm{N}( \mathbf{y}_{M-s:M} | L_{M-s} z_{M-s}, \Gamma_{M-s} ) \\
 &= \sqrt{\frac{|V_{M-s}|}{|\Gamma_{M-s}|}} \mathrm{N}(x_{M-s} | \nu_{M-s}, V_{M-s} )
\end{split}
\end{equation}
\begin{equation}
\begin{split}
 V_{M-s} & = (L_{M-s}^T \Gamma_{M-s}^{-1} L_{M-s})^{-1} \\
 \nu_{M-s} & = V_{M-s} L_{M-s}^T \Gamma_{M-s}^{-1} \mathbf{y}_{M-s:M}     ,
\end{split}
\end{equation}

where $\mathbf{y}_{M-s:M}$ is a vector of concatenated observations, and $L_{M-s}$ and $\Gamma_{M-s}$ may be constructed from the system models,
%
\begin{equation}
\begin{split}
\mathbf{y}_{M-s:M} & = \begin{bmatrix}y_M \\ y_{M-1} \\ \vdots \\ y_{M-L} \end{bmatrix}
\end{split}
\end{equation}
\begin{equation}
\begin{split}
L_{M-s} & = \begin{bmatrix} H_M \prod_{k=M-s+1}^{M} A_k \\ H_{M-1} \prod_{k=M-s+1}^{M-1} A_k \\ \vdots \\ H_{M-s+1} \end{bmatrix}
\end{split}
\end{equation}
\begin{equation}
\begin{split}
 \Gamma_{M-s} & = K_{M-s} \begin{bmatrix} Q_M & 0 & \dots & 0 \\ 0 & Q_{M-1} & \dots & 0 \\ \vdots & \vdots & \ddots & \vdots \\ 0 & 0 & \dots & Q_{M-s+1} \end{bmatrix} K_{M-s}^T \nonumber \\
  & + \: \begin{bmatrix} R_M & 0 & \dots & 0 \\ 0 & R_{M-1} & \dots & 0 \\ \vdots & \vdots & \ddots & \vdots \\ 0 & 0 & \dots & R_{M-s} \end{bmatrix} \nonumber
\end{split}
\end{equation}
\begin{equation}
\begin{split}
 K_{M-s} & = \begin{bmatrix} H_M & H_M A_M & \dots & H_M \prod_{k=M-s+2}^{M} A_k \\ 0 & H_{M-1} & \dots & H_{M-1} \prod_{k=M-s+2}^{M-1} A_k \\ \vdots & \vdots & \ddots & \vdots \\ 0 & 0 & \dots & H_{M-s+2} \\ 0 & 0 & \dots & 0 \end{bmatrix} \nonumber
\end{split}
\end{equation}

This inversion to initialise the backwards filter may be effected once $\Gamma_{M-s}$ becomes invertible. In the meantime, backward sampling may be conducted by using the joint observation density directly for calculation of the weights,
%
\begin{equation}
\begin{split}
  &\int p(y_{k+1:M} | u_{k+1:M}, z_k) p(z_k | u_{0:k}, y_{1:k}) dz_{k+1} \\
  &= \int \int p(y_{k+1:M} | u_{k+1:M}, z_{k+1}) p(z_{k+1} | u_{k+1}, z_{k}) dz_k p(z_k | u_{0:k}, y_{1:k}) dz_{k+1} \\
  &= \mathrm{N}( \mathbf{y}_{k+1:M} | \xi_{k+1} , \Lambda_{k+1} )     ,
\end{split}
\end{equation}
%
where,
%
\begin{equation}
\begin{split}
  \xi_{k+1} &= L_{k+1} A_{k+1} m_k \\
  \Lambda_{k+1} &= \Gamma_{k+1} + L_{L+1} ( A_{k+1} P_k A_{k+1}^T + Q_{k+1} ) L_{k+1}^T )     .
\end{split}
\end{equation}
%
By substituting in to weight formula, we obtain,
%
\begin{equation}
\begin{split}
  &w_{k|k+1}^{(i)} \propto w^{(i)}_k p(u_{k+1}\,|\,u_{k}^{(i)}) \mathrm{N}( \mathbf{y}_{k+1:M} | \xi_{k+1} , \Lambda_{k+1} )     .
\end{split}
\end{equation}



\subsection{Computational Complexity}

\subsubsection{Backward Simulation Smoothers}

Although all the backwards simulation smoothers technically sample $u_{1:k}$ at every step, all but $u_k$ are then marginalised again. Therefore, it is not necessary to store all the filter/smoother results from every time frame. This limits the required memory to $O(MN)$, linear in both the number of particles and the number of time steps.

In order to conduct backwards simulation smoothing for one time step of one trajectory, it is necessary to calculate the weight, $w_{k|k+1}^{(i)}$ for each of the $N$ filtering trajectories. Hence, to simulate $S$ trajectories, the computational complexity is $O(NMS)$. If we desire $S$ to be large, of the same order as $N$, then these methods are quadratic in the number of particles, which is limiting.

Note that while both the linear sampling and fully Rao-Blackwellised smoothers both have the same complexity scaling, the fully Rao-Blackwellised form is still faster by a scale factor. This is because the weight calculations require only one backwards Kalman filter step to be run for the sampled future, while the linear sampling version requires $N$ forward Kalman filter predictions to be made.

\subsubsection{Reweighting smoother}

The reweighting smoother only requires the latest nonlinear state to be stored from the filtering distribution, so the required memory is only $O(MN)$, the same as the simulation smoothers. The computational complexity is the same as that for the non-Rao-Blackwellised case, at $O(MN^2)$.



\subsection{Reduction of Computational Complexity}

The utility of smoothing algorithms is sometimes limited by their high computational complexity. Methods exist to reduce the complexity of backward simulation smoothers. Here we extend these methods for use with Rao-Blackwellised smoothers.

The limiting step in all the backward simulation smoothers is the calculation of the $N$ particle weights at every step used to sample from $p(u_{0:k}|u_{k+1:M}, y_{1:M})$ or $p(u_{0:k}|u_{k+1:M}, z_{k+1:M}, y_{1:M})$. These weights are sampled once and then discarded. A saving can be made by sampling the distribution indirectly using either rejection sampling or Metropolis-Hastings so that not all the weights need be calculated.

The method proposed by \cite{Douc+Garivier+Moulines+Olsson:2010} and applied by \cite{Lindsten+Schon:2011} to the linear sampling Rao-Blackwellised smoother uses rejection sampling \comment{Cite something describing RS?}. A state history is first proposed using the RBPF approximation as a proposal,
%
\begin{equation}
\begin{split}
  u_{0:k}^* \sim p(u_{0:k} | y_{1:k}) \approx \sum_i w_k^{(i)} \delta( u_{0:k} - u_{0:k}^{(i)} )     .
\end{split}
\end{equation}

The sample is then accepted with probability,
%
\begin{equation}
\begin{split}
  \frac{ w_{k|k+1}^* }{ w_k^* \eta_k }     ,
\end{split}
\end{equation}
%
where $\eta_k$ is a bound on the maximum value of $\frac{ w_{k|k+1}^* }{ w_k^* }$. In the linear sampling case, this is limited by
%
\begin{equation}
\begin{split}
  \eta_k &\ge \frac{ w_{k|k+1}^* }{ w_k^* } \\
  & = \max_i( p(\tilde{u}_{k+1} | u_k^{(i)}) \mathrm{N}(\tilde{z}_{k+1}\,|\, m^{-(i)}_{k+1}, P^{-(i)}_{k+1}) ) \\
  & \ge \max_{u_k}( p(\tilde{u}_{k+1} | u_k^{(i)}) ) \max_{z_{k+1}}( \mathrm{N}(\tilde{z}_{k+1}\,|\, m^{-}_{k+1}, P^{-}_{k+1}) )   .
\end{split}
\end{equation}

\comment{Oooh er! This doesn't work! With the freedom to vary $m^-$ and $P^-$, the Gaussian is unbounded, which means the only way to bound the target/proposal ratio is to exhaustively calculate the ratio for all particles, which is exactly what we're trying to avoid.}

In the full Rao-Blackwellised case,
%
\begin{equation}
\begin{split}
  \eta_k &\ge \frac{ w_{k|k+1}^* }{ w_k^* } \\
  & = \max_i( |\det A(u_{k+1})|^{-1} \mathrm{N}(m_k^{(i)}\,|\,m^{-b}_{k},P_k^{(i)}+P^{-b}_{k}) \, p(u_{k+1}\,|\,u_{k}^{(i)}) ) 
\end{split}
\end{equation}


Using rejection sampling reduces the computational complexity to $O(RMS)$ where $R$ is the average number of rejections before a sample is accepted over the whole procedure. Thus the operation time is reduced when $R < N$. Numerical tests show that this is not always the case. \cite{Lindsten+Schon:2011} introduce a timeout operation on the sampling to mitigate the effect of excessive numbers of rejections.

An alternative method is to use Metropolis-Hastings for sampling $p(u_{0:k}|u_{k+1:M}, y_{1:M})$. This was introduced in \cite{Bunch+Godsill:2012}. As in the rejection sampling method, histories are proposed using the filtering weights and accepted with a probability given by the ratios of target and proposal weights for the previous, $u_{0:k}$, and new, $u_{0:k}^*$, history. In the linear sampling case,
%
\begin{equation}
\begin{split}
  \alpha &= \frac{ p(u_{0:k}^*|u_{k+1:M}, z_{k+1:M}, y_{1:M}) }{ p(u_{0:k}^*|y_{1:k}) }\frac{ p(u_{0:k}|y_{1:k}) }{ p(u_{0:k}|u_{k+1:M}, z_{k+1:M}, y_{1:M}) } \\
  &= \frac{ p(\tilde{u}_{k+1} | u_k^{*}) \mathrm{N}(\tilde{z}_{k+1}\,|\, m^{-*}_{k+1}, P^{-*}_{k+1}) ) }{ p(\tilde{u}_{k+1} | u_k) \mathrm{N}(\tilde{z}_{k+1}\,|\, m^{-}_{k+1}, P^{-}_{k+1}) ) }     .
\end{split}
\end{equation}
%
In the fully Rao-Blackwellised case,
%
\begin{equation}
\begin{split}
  \alpha &= \frac{ p(u_{0:k}^*|u_{k+1:M}, y_{1:M}) }{ p(u_{0:k}^*|y_{1:k}) }\frac{ p(u_{0:k}|y_{1:k}) }{ p(u_{0:k}|u_{k+1:M}, y_{1:M}) } \\
  &= \frac{ \mathrm{N}(m_k^{*}\,|\,m^{-b}_{k},P_k^{*}+P^{-b}_{k}) \, p(u_{k+1}\,|\,u_{k}^{*}) }{ \mathrm{N}(m_k\,|\,m^{-b}_{k},P_k+P^{-b}_{k}) \, p(u_{k+1}\,|\,u_{k}) }     .
\end{split}
\end{equation}

The chain may be initialised using the history selected in the previous sampling step (which would normally be marginalised). The number of steps, $C$, in the Markov chain may be chosen.

When using this Metropolis-Hastings sampling, the memory requirements are higher, because it is necessary to store both $u_{k-1}$ and $u_k$ for each particle in the filter approximation. The computational complexity is reduced to $O(CMS)$. $C$ may be chosen by design to be lower than $N$. If the acceptance probabilities are low, then the particle diversity of the approximation produced will be poor. (In the worst case, when only one particle has a non-zero weight, $w_{k|k+1}$, then this algorithm will simply reproduce particles from the filter/smoother approximation).



\subsection{State Dependent Latent Variable Dynamics}

\comment{Including the state to latent variable equation as in the
  methods of Lindsten \& co.}

\subsection{Approximate Rao-Blackwellization}

\comment{Approximate Rao-Blackwellization by replacing Kalman/RTS
with their non-linear counterparts.}




\section{Theoretical Analysis}

\subsection{Analysis of Backward Simulation}

\comment{Proof that the methods samples from the right distribution}

\comment{Proof that the error of this backward simulation
is smaller than in the approaches of Fong and Lindsten/Sch\"on}

\subsection{Analysis of Reweighting}

\comment{Proof that the method samples with the right distribution
in the limit of infinite number of particles (if it does)}

\comment{Analyze the approximation with finite number of particles}

\subsection{Analysis of Kim's Approximation}

\comment{Theoretical analysis of accuracy as compared to the full
  approach}


\section{Numerical Results}

\subsection{Simple Switching Linear Model}
%
\comment{We could leave this too simple simulation out?}
For testing the practical performance of the proposed algorithm, we
used a model in which the linear Gaussian part of the state $z_k$ is
2-dimensional and the non-Gaussian part is an indicator variable taking
discrete values $c_k \in\{1,2\}$. The model is the following:
%
\begin{itemize}
\item If the latent variable $c_{k-1} = 1$, then the dynamic model for
  the step $t_{k-1} \rightarrow t_k$ is the Wiener velocity model
  (cf. \cite{Bar-Shalom+Li+Kirubarajan:2001}) $\ddot{z} = w(t)$,
  where $w(t)$ is a white noise process with spectral density $q_c =
  0.01$
\item If the latent variable $c_{k-1} = 2$ then the dynamic model
  is a damped harmonic force model $\ddot{z} = -z/10 - \dot{z}/10 + w(t)$.
\item If $c_k = 1$ then we measure $z$ with variance $1$, otherwise
  with variance $5^2$.
\end{itemize}
%
The transition probabilities for $c_k$ were $\Pi_{1\,|\,1} = 0.8$,
$\Pi_{1\,|\,2} = 0.2$, $\Pi_{2\,|\,1} = 0.2$, and $\Pi_{2\,|\,2} =
0.8$. The initial distributions were $P(c_0=1) = 0.9$, $P(c_0=2) =
0.1$ and $z_0 \sim \mathrm{N}((0~1)^T,I)$. The sampling period was
$\Delta t = 0.1$ and the continuous time models were discretized using
the standard procedure (see \cite{Bar-Shalom+Li+Kirubarajan:2001}).
The ``optimal importance distribution''
\cite{Doucet+Godsill+Andrieu:2000} was used as a proposal for the
filter.

The following methods were tested:
%
\begin{itemize}
\item {\em RBPF}: Rao-Blackwellized particle filter.
\item {\em KiS}: Rao-Blackwellized \cite{Kitagawa:1996} smoother.
\item {\em KGS}: Kim's approximation based Rao-Blackwellized
  \cite{Godsill+Doucet+West:2004} smoother.
\item {\em FGS}: \cite{Fong+Godsill+Doucet+West:2002} smoother.
\item {\em RBG}: The proposed Rao-Blackwellized
  \cite{Godsill+Doucet+West:2004} smoother in Algorithm
  \ref{alg:rbbssmooth}.
\end{itemize}
%
Performance was assessed by comparison of the root mean square error
in the linear part of the state ({\em RMSE}). In addition, at each
time step a maximum a posteriori estimate for the value of the
indicator variable was made by selecting the option with the greater
weight. The proportion of errors in these estimates was then
calculated ({\em ErrRate}). Furthermore, by averaging the total weight
assigned to the unselected indicator values, we can calculate the
error rate predicted by each algorithm ({\em PredRate}). A comparison
of the predicted and observed error rates gives an additional measure
of how accurately the algorithms characterize the posterior
distribution.

The results from 1000 independent simulation runs using 100 particles
in each of the methods are shown in Table \ref{tbl:res100}. The RMSE
results of the smoother methods are almost equal, except KiS gives a slightly
higher RMSE on average than the other smoothers. The error rate of KiS
is higher than of the other smoothers although the predicted error
rate is significantly lower than of the others. The RMSEs, error rates
and predicted errors of the smoothers as a function of time are shown in
Figures \ref{fig:100_rmse}--\ref{fig:100_errp}. It can be seen that
RMSE and error rate of KiS increases (relative to other methods), and
the predicted error rate decreases toward the beginning of the data.
These effects are likely to be due to the degeneracy of KiS. The error
rate and predicted error rate of KGS are slightly higher than those of FGS
and RBG, whereas the results of the latter seem to be practically
identical.

%Results from rbpfs1-results-20111021/.
%Number of MC simulations = 1000.
%Smoother MC samples N=100 KGN=100 RBG1N=100 FGN=100.
%Method RMSE              ErrRate           PredRate
%RBPF   0.49292 (0.00313) 0.16462 (0.00060) 0.15986 (0.00021)
%KiS    0.27001 (0.00178) 0.15798 (0.00072) 0.08953 (0.00052)
%KGS    0.26364 (0.00176) 0.13546 (0.00063) 0.13065 (0.00027)
%FGS    0.26292 (0.00174) 0.12843 (0.00062) 0.12348 (0.00028)
%RBG    0.26188 (0.00173) 0.12864 (0.00061) 0.12345 (0.00027)
%
\begin{table}[htb!]
\begin{center}
\begin{tabular}{|c|c|c|c|c|}
\hline
Method & RMSE              & ErrRate       &   PredRate \\
\hline
\hline
RBPF   & 0.49 & 0.16 & 0.16 \\ \hline
KiS    & 0.27 & 0.16 & 0.09 \\ \hline
KGS    & 0.26 & 0.14 & 0.13 \\ \hline
FGS    & 0.26 & 0.13 & 0.12 \\ \hline
{\bf RBG} & {\bf 0.26} & {\bf 0.13} & {\bf 0.12} \\ \hline
\end{tabular}
\end{center}
\caption{Simulation results with 100 particles averaged over 1000 runs. }
\label{tbl:res100}
\end{table}

\begin{figure}[htb!]
\begin{center}
%\includegraphics[width=0.45\textwidth]{fig100_rmse}
\end{center}
\caption{RMSEs with 100 particles averaged over 1000 runs.}
\label{fig:100_rmse}
\end{figure}

\begin{figure}[htb!]
\begin{center}
%\includegraphics[width=0.45\textwidth]{fig100_err}
\end{center}
\caption{Error rates with 100 particles averaged over 1000 runs.}
\label{fig:100_err}
\end{figure}

\begin{figure}[htb!]
\begin{center}
%\includegraphics[width=0.45\textwidth]{fig100_errp}
\end{center}
\caption{Predicted error rates with 100 particles averaged over 1000
  runs.}
\label{fig:100_errp}
\end{figure}

%Results from rbpfs1-results-20111020b/.
%Number of MC simulations = 1000.
%Smoother MC samples N=10 KGN=10 RBG1N=10 FGN=10.
%Method RMSE              ErrRate           PredRate
%RBPF   0.51773 (0.00352) 0.17858 (0.00064) 0.14359 (0.00023)
%KiS    0.29946 (0.00214) 0.18350 (0.00070) 0.01824 (0.00038)
%KGS    0.28782 (0.00206) 0.15695 (0.00066) 0.10810 (0.00027)
%FGS    0.29621 (0.00211) 0.15174 (0.00065) 0.10037 (0.00028)
%RBG    0.28610 (0.00203) 0.15170 (0.00064) 0.10056 (0.00027)
%
\begin{table}[htb]
\begin{center}
\begin{tabular}{|c|c|c|c|c|}
\hline
Method & RMSE              & ErrRate       &   PredRate \\
\hline
\hline
RBPF   & 0.52 & 0.18 & 0.14 \\ \hline
KiS    & 0.30 & 0.18 & 0.02 \\ \hline
KGS    & 0.29 & 0.16 & 0.11 \\ \hline
FGS    & 0.30 & 0.15 & 0.10 \\ \hline
{\bf RBG} & {\bf 0.29} & {\bf 0.15} & {\bf 0.10} \\ \hline
\end{tabular}
\end{center}
\caption{Simulation results with 10 particles averaged over 1000 runs. }
\label{tbl:res10}
\end{table}

To test the effect of using a small number of particles, we also ran
1000 simulation runs with only 10 particles. The results are shown in
Table \ref{tbl:res10} and they seem to be consistent with the 100
particle results. Now the RMSE of FGS is slightly higher on average
than that of RBG. This is to be expected, because the linear state in
FGS is now represented by 10 samples whereas RBG maintains the whole
Gaussian distribution. The higher RMSE of FGS can also be clearly seen
in the Figure \ref{fig:10_rmse}.

\begin{figure}[htb]
\begin{center}
%\includegraphics[width=0.45\textwidth]{fig10_rmse}
\end{center}
\caption{RMSEs with 10 particles averaged over 1000 runs.}
\label{fig:10_rmse}
\end{figure}

\begin{figure}[htb]
\begin{center}
%\includegraphics[width=0.35\textwidth]{fig10_err}
\end{center}
\caption{Error rates with 10 particles averaged over 1000 runs.}
\label{fig:10_err}
\end{figure}

\begin{figure}[htb]
\begin{center}
%\includegraphics[width=0.35\textwidth]{fig10_errp}
\end{center}
\caption{Predicted error rates with 10 particles averaged over 1000
  runs.}
\label{fig:10_errp}
\end{figure}


\subsection{High-Dimensional Switching Linear Model}

\comment{This could be the varying dimensional linear model and the
point is to compare to other methods}

\subsection{Audio Signal Analysis}

\comment{Maybe we could demonstrate some application without actually
attempting to compare to other methods?}


\section{Conclusion and Discussion}

\comment{Conclusion and discussion here}

%\subsection{Conclusion}


\begin{ack}                               % Place acknowledgements
%Partially supported by the Roman Senate.  % here.
\comment{Acknowledgements here}
\end{ack}

\bibliographystyle{plain}        % Include this if you use bibtex
\bibliography{rb-smoother}       % and a bib file to produce the
                                 % bibliography (preferred). The
                                 % correct style is generated by
                                 % Elsevier at the time of printing.

%\begin{thebibliography}{99}     % Otherwise use the
                                 % thebibliography environment.
                                 % Insert the full references here.
                                 % See a recent issue of Automatica
                                 % for the style.
%  \bibitem[Heritage, 1992]{Heritage:92}
%     (1992) {\it The American Heritage.
%     Dictionary of the American Language.}
%     Houghton Mifflin Company.
%  \bibitem[Able, 1956]{Abl:56}
%     B.~C.~Able (1956). Nucleic acid content of macroscope.
%     {\it Nature 2}, 7--9.
%  \bibitem[Able {\em et al.}, 1954]{AbTaRu:54}
%     B.~C. Able, R.~A. Tagg, and M.~Rush (1954).
%     Enzyme-catalyzed cellular transanimations.
%     In A.~F.~Round, editor,
%     {\it Advances in Enzymology Vol. 2} (125--247).
%     New York, Academic Press.
%  \bibitem[R.~Keohane, 1958]{Keo:58}
%     R.~Keohane (1958).
%     {\it Power and Interdependence:
%     World Politics in Transition.}
%     Boston, Little, Brown \& Co.
%  \bibitem[Powers, 1985]{Pow:85}
%     T.~Powers (1985).
%     Is there a way out?
%     {\it Harpers, June 1985}, 35--47.

%\end{thebibliography}

%\appendix
%\section{A summary of Latin grammar}    % Each appendix must have a short title.
%\section{Some Latin vocabulary}         % Sections and subsections are supported
                                        % in the appendices.

\end{document} 